\chapter{Technische Informationen}
\label{ch:TechnischeInfos}

\newcommand*{\checkbox}{{\fboxsep 1pt%
\framebox[1.30\height]{\vphantom{M}\checkmark}}}

\section{Aktuelle Dateiversionen}

\begin{center}
\begin{tabular}{|l|l|}
\hline
Datum & Datei \\
\hline\hline
\hgbthesisDate & \texttt{hgbthesis.cls} \\
\hline
\hgbDate       & \texttt{hgb.sty} \\
\hline
\end{tabular}
\end{center}




\section{Details zur aktuellen Version}


Das ist eine völlig überarbeitete Version der DA/BA-Vorlage, die
\mbox{UTF-8} kodierten Dateien vorsieht und ausschließlich im PDF-Modus arbeitet.
Der "`klassische"' DVI-PS-PDF-Modus wird somit nicht mehr unterstützt! 

\subsection{Allgemeine technische Voraussetzungen}

Eine aktuelle \latex-Installation mit
\begin{itemize}
	
		\item Texteditor für \mbox{UTF-8} kodierte (Unicode) Dateien,
		\item \texttt{biber}-Programm (BibTeX-Ersatz, Version $\geq 1.5$),
		\item \texttt{biblatex}-Paket (Version $\geq 2.5$, 2013/01/10),
		\item Latin Modern Schriften (Paket \texttt{lmodern}).%
			\footnote{\url{http://www.ctan.org/pkg/lm}, \url{http://www.tug.dk/FontCatalogue/lmodern}}
\end{itemize}


\subsection{Verwendung unter Windows}

Eine typische Installation unter Windows sieht folgendermaßen aus
(s.\ auch Abschnitt \ref{sec:Windows}):
%
\begin{enumerate}
\item \textbf{MikTeX 2.9}%
	\footnote{\url{http://www.miktex.org/} -- \textbf{Achtung:} 
	Generell wird die \textbf{Komplett\-installation} von MikTeX ("`Complete MiKTeX"') empfohlen, 
	da diese bereits alle notwendigen Zusatzpakete und Schriftdateien enthält! 
	Bei der Installation ist darauf zu achten, 
	dass die automatische Installation erforderlicher Packages 
	durch "`\emph{Install missing packages on-the-fly: = Yes}"' ermöglicht wird (NICHT "`\emph{Ask me first}"')!
	Außerdem ist zu empfehlen, unmittelbar nach der Installation von MikTeX mit dem Programm
	\texttt{MikTeX} $\to$ \texttt{Maintenance} $\to$ \texttt{Update} und \texttt{Package Manager} 
	ein Update der installierten Pakete durchzuführen.}
	(zurzeit am einfachsten die 32-Bit Version, da nur diese das Programm \texttt{biber.exe} 
	bereits enthält),
\item \textbf{TeXnicCenter 2.0}%
	\footnote{\url{http://www.texniccenter.org/}}
	(Editor-Umgebung, unterstützt UTF-8),
\item \textbf{SumatraPDF}%
	\footnote{\url{http://blog.kowalczyk.info/software/sumatrapdf/}} 
	(PDF-Viewer),
\end{enumerate}
%
Ein passendes TeXnicCenter-Profil für MikTeX, Biber und Sumatra ist in diesem Paket enhalten
(Datei \verb!_tc_output_profile_sumatra_utf8.tco!). Dieses sollte man zuerst
über \texttt{Build} $\to$ \texttt{Define Output Profiles} in TeXnicCenter importieren.
\textbf{Achtung}: Alle neu angelegten \texttt{.tex}-Dateien sollten in UTF-8 Kodierung gespeichert werden!




\subsection{Verwendung unter Mac~OS}


Diese Version sollte insbesondere mit \emph{MacTeX} problemlos laufen (s.\ auch Abschnitt \ref{sec:MacOs}):
\begin{enumerate}
\item 
	\emph{MacTex} (2012 oder höher).
\item 
	Die Zeichenkodierung des Editors sollte auf UTF-8 eingestellt sein.
\item 
	Als Engine (vergleichbar mit den Ausgabeprofilen in TeXnicCenter) sollte \emph{LaTeXMk} verwendet werden. 
	Dieses Perl-Skript erkennt automatisch, wie viele Aufrufe von \emph{pdfLaTeX} und \emph{Biber} nötig sind. 
	Die Ausgabeprofile \emph{LaTeX} oder \emph{pdfLaTeX} hingegen müssen mehrmals aufgerufen werden, 
	zudem werden hierbei auch die Literaturdaten nicht verarbeitet. Dazu müsste extra die \emph{Biber}-Engine 
	aufgerufen werden, 	die jedoch noch nicht in allen Editoren vorhanden ist.
\end{enumerate}


\begin{comment}
\subsection{Vorteile}
\begin{itemize}
\item PDF wird direkt erzeugt ohne DVI und PS; damit ist angeblich auch die "`Feintypographie"' besser.
\item Die Verwendung von \texttt{SumatraPDF} erlaubt funktionierende Forward- und Inverse-Suche, womit erstmals ein effektiver PDF-Workflow möglich ist.
\item Preview der vollständigen Manuskripts (inklusive Grafiken) ist in PDF viel schneller
als in DVI (mit YAP und Ghostscript für die Grafiken).
\item Grafiken können auch als PDF, PNG oder JPEG direkt eingebunden werden. Bestehende EPS-Grafiken werden automatisch in PDF konvertiert. 
\item Bei eingebundenen Rasterbildern werden (im Unterschied zu \texttt{ps2pdf} in der Default-Einstellung) keine zusätzlichen JPEG-Artefakte erzeugt. 
(Anmerkung: im TC-Ausgabeprofil für \texttt{ps2pdf} ist dafür jetzt die
Option \verb!-dPDFSETTINGS=/prepress! eingestellt -- \verb!=/printer! ist nicht ausreichend!)
\item Die Erzeugung von aktiven Verweisen mit \texttt{hyperref} funktioniert problemlos, mit allen Vorteilen (einschließlich der Zeilenumbrüche in URLs).
\item PDF-Metadaten (zur verbesserten Suche) werden direkt aus den Dokumentendaten durch LaTeX generiert.
\end{itemize}

\subsection{Weitere Neuerungen}
%
\begin{sloppypar}
\begin{itemize}
\item Verwendung des \texttt{epstopdf}-Pakets, wodurch vorhandene EPS-Grafiken (mit denen \texttt{pdflatex} nicht umgehen kann) automatisch in PDF-Dateien konvertiert werden, unter der Annahme, dass \texttt{epstopdf.exe} vorhanden ist. Das ist bei Rasterbildern allerdings nicht zu enpfehlen, weil mit \texttt{epstopdf} die Kompressionsqualität nicht gesteuert erden kann. In diesem Fall ist es besser, die EPS-Dateien (\zB\ mit PhotoShop) direkt in PDFs zu konvertieren oder (noch besser) die Original JPEG- oder PNG-Dateien zu verwenden.
%
\item Unter \texttt{pdflatex} können nun (mit \verb!\includegraphics{}!) neben PDFs auch Bilder im JPEG- oder PNG-Format direkt eingebunden werden. Alle Datei-Extensions der Grafikdateien wurden im Quelltext entfernt.
%
\item 
Verwendung des \textbf{SumatraPDF}-Viewers anstelle von Adobe Acrobat, da Acrobat das Überschreiben der Ausgabedatei blockiert (unter Windows) und forward/inverse Suche schlecht \bzw\ gar nicht unterstützt.
Anweisungen zur Einstellung findet man unter \url{http://www.hehn.biz/Mar/How_to_Sumatra.pdf} -- diese sind auch im beiliegenden TC-Aus\-gabe\-profil implementiert.
%
\item Verwendung des \texttt{pdfsync}-Pakets zur Unterstützung der inversen Suche aus PDF-Dateien.
%
\item Verwendung des \texttt{hyperref}-Pakets zur Aktivierung von Links (Web, Inhaltsverzeichnis, Querverweise, Literatur etc.). Erzeugt auch eine Navigation-Pane.
%
\item PDF-Metadaten werden automatisch aus den Dokumentendaten generiert (durch \texttt{hyperref} möglich).
%
\item Verwendung des \texttt{breakurl}-Pakets, mit dem Zeilenumbrüche trotz \texttt{hy\-per\-ref} auch bei DVI-PS-PDF-Generierung durchgeführt werden. Dadurch sind jetzt auch URLs in Captions und Fußnoten problemlos möglich und auch \verb!\urldef{}! ist nicht mehr erforderlich (entspr.\ Textpassagen in \ref{sec:QuellenangabenInCaptions} entfernen!). 
%
\item Alle bestehenden EPS-Dateien mit Rasterbildern wurden auf Binärkodierung umgestellt, da dies mit der aktuellen MikTeX-Version keine Probleme mehr verursacht. Zusätzlich wurden PNG-Versionen für \texttt{pdflatex} angelegt, sodass keine automatische Umwandlung mit \texttt{epstopdf} erfolgt.
%
\item
Das lästige Problem des übermäßigen vertikalen Abstände in LaTeX-Aufzählungslisten wurde mit dem \texttt{enumitem}-Paket behoben. Alle \verb!\itemsep0pt! Anweisungen im Text wurden entfernt.
%
\item Einbindung des \texttt{cite}-Pakets mit \texttt{noadjust}-Option, womit kein zusätzliches Spacing erzeugt wird.
\end{itemize}
\end{sloppypar}
\end{comment}


\begin{comment}
\section{Einstellungen unter Windows} 
\label{sec:EinstellungAusgabeprofile}

Die folgenden Angaben beziehen sich auf eine bewährte Arbeitsumgebung unter Windows (XP, Win7) mit MikTeX, Sumatra-PDF und TeXnicCenter, mit folgenden Installationspfaden:
%
\begin{quote}
\verb!C:\Program Files (x86)\MiKTeX 2.9\! \\
\verb!C:\Program Files (x86)\SumatraPDF\! \\
\verb!C:\Program Files (x86)\TeXnicCenter\! 
\end{quote}
%
Unter Windows XP liegen die Programme in \verb!C:\Program Files\!.
Falls neuere Versionen dieser Komponenten installiert sind, müssen natürlich die nachfolgend angegebenen Pfade entsprechend modifiziert werden.

\begin{quote}
\textbf{Achtung:} Für MikTeX immer die \textbf{komplette Version} installieren! Das entsprechende Installationsverzeichnis hat aktuell einen Umfang von ca.\ 1.2 GB und enthält etwa 53.200 Dateien 
(typischerweise in \nolinkurl{C:\\Program Files (x86)\\MiKTeX...}).
\end{quote}
\end{comment}

\begin{comment}
\subsection{TeXnicCenter-Ausgabeprofile}
\label{sec:TeXnicCenterUndMikTeX}

TeXnicCenter definiert den Verarbeitungsablauf des LaTeX-Dokuments anhand von Ausgabeprofilen, wobei die oben genannten Komponenten als externe Programme mit entsprechenden Argumenten aufgerufen werden.
Die Einstellung der Ausgabeprofile erfolgt in TeXnicCenter über das Menü
\textsf{Ausgabe}$\rightarrow$\textsf{Ausgabeprofile definieren...} (Abb.\ \ref{fig:techniccenter-profile-latex}). 
Die Profile werden (abhängig von der installierten Software) üblicherweise beim ersten Start von TeXnicCenter durch den zugehörigen "`Wizard"' voreingestellt. 

\begin{figure}
\centering\small
\setlength{\tabcolsep}{0pt}%
\begin{tabular}{c@{~}c}
\includegraphics[width=0.49\textwidth]{techniccenter-profile-dvi-26} &
\includegraphics[width=0.49\textwidth]{techniccenter-profile-dvips-26} \\[4pt]
(a) & (b)
\end{tabular}
\caption{Spezifikation der Ausgabeprofile in TeXnicCenter.}
\label{fig:techniccenter-profile-latex}
\end{figure}

In der Datei \verb!tc_output_profiles_sumatra.tco! sind  folgende beiden "`maßgeschneiderten"' Ausgabeprofile für TexNicCenter angelegt (Import über \textsf{Build} $\rightarrow$ \textsf{Define Output Profiles ...}):
\begin{itemize}
	\item \verb!LaTeX => PDF (Sumatra)! -- Standard, direkte Erzeugung von PDF,
	\item \verb!LaTeX => PS => PDF (Sumatra)! -- PDF "`klassisch"' via DVI und PS.
\end{itemize}

\subsubsection{Profil "`\texttt{LaTeX => PDF (Sumatra)}"'}

Das ist das mit diesem Setup normalerweise verwendete Standardprofil.

\paragraph{(La)Tex:}
\begin{itemize}
  \item Path to the (La)TeX compiler: \\
        \begin{small} \verb!C:\Program Files (x86)\MiKTeX 2.9\miktex\bin\pdflatex.exe!\end{small}
  \item Command line arguments to pass to the compiler:\\
\begin{small}
   \verb!-synctex=-1 -interaction=nonstopmode "%pm"!
\end{small}
\end{itemize}

\paragraph{Postprocessor:} 
leer, kein Postprocessor notwendig.

\paragraph{Viewer:}
\begin{itemize}
\item Path of executable: \\
\begin{small}
    \verb!C:\Program Files (x86)\SumatraPDF\SumatraPDF.exe ! \\ 
    \verb!-inverse-search "\"C:\Program Files\TeXnicCenter\TEXCNTR.EXE\" !\\
    \verb!/ddecmd \"[goto('%f','%l')]\""!
\end{small}
%
\item View project's output: \\
\begin{small}
    \checkbox\ Command line argument \\\
    Command: \verb!"%bm.pdf"!
\end{small}
%
\item Forward search:\\
\begin{small}
    \checkbox\ DDE command \\\
    Command: \verb![ForwardSearch("%bm.pdf","%Wc",%l,0)]! \\
    Server: \verb!SUMATRA! \\
    Topic: \verb!Control!
\end{small}
\item Close document before running (La)TeX:\\
\begin{small}
    \checkbox\ Do not close
\end{small}
\end{itemize}


\subsubsection{Profil "`\texttt{LaTeX => PS => PDF (Sumatra)}"'}

Profil ausschließlich für den DVI-PS-Workflow (über DVI und PostScript).

\paragraph{(La)Tex:}
\begin{itemize}
  \item Path to the (La)TeX compiler: \\
        \begin{small} \verb!C:\Program Files (x86)\MiKTeX 2.9\miktex\bin\latex.exe!\end{small}
  \item Command line arguments to pass to the compiler:\\
\begin{small}
   \verb!-synctex=-1 -interaction=nonstopmode "%pm"!
\end{small}
\end{itemize}

\paragraph{Postprocessor:}
\begin{itemize}
  \item DviPS (PDF): \\
        \begin{small} 
        Executable: \verb!C:\Program Files (x86)\MiKTeX 2.9\miktex\bin\dvips.exe! \\
        Arguments: \verb!-ta4 -P pdf -R0 "%Bm.dvi"!
        \end{small}
  \item Ghostscript (ps2pdf):\\
  		\begin{small} 
        Executable: \verb!C:\Program Files (x86)\gs\gs9.04\bin\gswin32c.exe! \\
        Arguments: \verb!-q -dPDFSETTINGS=/prepress -sPAPERSIZE=a4 -dSAFER! \\
         \verb!-dBATCH -dNOPAUSE -sDEVICE=pdfwrite -sOutputFile="%bm.pdf"! \\
         \verb!-c save pop -f "%bm.ps"!
      \end{small}
\end{itemize}

\paragraph{Viewer:}
wie in Profil A. (\texttt{LaTeX => PDF (Sumatra)}).

\section{Tipps und offene Probleme:}

\begin{itemize}
\item \texttt{psfrag} funktioniert nicht mit \texttt{pdflatex} und es gibt auch leider keine Ersatzlösung. 
Wenn man \texttt{psfrag} braucht, dann muss man weiterhin über PostScript 
(\verb!LaTeX => PS => PDF!) arbeiten (was allerdings nunmehr auch mit \texttt{hyperref} kein Problem mehr ist).
%
\item Bei Verwendung des TexWorks-Editors (wird mit MikTeX ausgeliefert) sollte man die Standard-Zeichenkodierung von \emph{Unicode} (utf8) auf \emph{Latin-1} (ISO 8859-1) umstellen.
%
\item Adobe Illustrator kann beim Speichern als PDF die Bounding Box nicht setzen. 
Eine Möglichkeit ist, die Grafik zuerst als EPS zu exportieren und dann mit Acrobat in ein PDF zu konvertieren. 
%
\end{itemize}
\end{comment}



\begin{comment}
\section{Einstellungen für YAP (DVI-Viewer) im DVI-PS-Workflow}
\label{sec:YapEinstellung}

Im Standard-DVI-Viewer YAP lässt sich durch Mausklick auf das DVI-Dokument sehr leicht die zugehörige Stelle im Quelltext finden. Im Normalfall öffnet dann TeXnicCenter das zugehörige \latex-Dokument automatisch an der richtigen Stelle.
Das zugehörige "`Inverse DVI Search"' Kommando sollte sich bereits bei der Installation richtig einstellen.

Falls dies \emph{nicht} funktioniert, kann man in YAP diese Einstellung auch manuell über das Menü \textsf{View}\thinspace$\rightarrow$\thinspace\textsf{Options...} vornehmen, wie in Abb.\ \ref{fig:yap-inverse-search} gezeigt.
In diesem Fall lautet die vollständige Anweisung in "`Command Line"' folgendermaßen:
\begin{center}\footnotesize
\verb!"C:\Program Files (x86)\TeXnicCenter\TEXCNTR.EXE" /ddecmd "[goto('%f', '%l')]"!
\end{center}


\begin{figure}
\centering\small
\includegraphics[width=1.0\textwidth]{yap-inverse-search-settings}
\caption{"`Inverse DVI Search"' Einstellung in YAP (über das Menü \textsf{View}\thinspace$\rightarrow$\thinspace\textsf{Options...}).}
\label{fig:yap-inverse-search}
\end{figure}

Latex
C:\Program Files\MiKTeX 2.6\miktex\bin\latex.exe
--src -interaction=nonstopmode "%Wm"

Bibtex
C:\Program Files\MiKTeX 2.6\miktex\bin\bibtex.exe
"%bm"

---

DviPs (PDF)
C:\Program Files\MiKTeX 2.6\miktex\bin\dvips.exe
-ta4 -Ppdf  -R0 "%Bm.dvi"

Ghostscript (ps2pdf)
C:\Program Files\gs\gs8.61\bin\gswin32c.exe
-sPAPERSIZE=a4 -dSAFER -dBATCH -dNOPAUSE -sDEVICE=pdfwrite -dPDFSETTINGS=/prepress -sOutputFile="%bm.pdf" -c save pop -f "%bm.ps"


YAP 
Options -> Inverse Search
"C:\Program Files\TeXnicCenter\TEXCNTR.EXE" /ddecmd "[goto('%f', '%l')]"

\end{comment}

