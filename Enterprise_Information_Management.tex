\documentclass[praktikum,german]{hgbthesis}

\RequirePackage[onehalfspacing]{setspace}
\RequirePackage{hyperref}
\RequirePackage{caption}
\RequirePackage[utf8]{inputenc}
\RequirePackage[german]{babel}
\RequirePackage[nottoc]{tocbibind}
\RequirePackage[onehalfspacing]{setspace}

\graphicspath{{images/}}
\logofile{logo}
\bibliography{literatur}

\DefineBibliographyStrings{german}{%
	urlseen = {eingsehen am},
}

%%%----------------------------------------------------------
\begin{document}
%%%----------------------------------------------------------

\author{Stefan Trageiser}
\studienort{München}
\studiengang{Wirtschaftsinformatik}

\abgabedatum{2017}{01}{01}



%%%----------------------------------------------------------
\frontmatter
\maketitle

\tableofcontents
%%%----------------------------------------------------------


%%%----------------------------------------------------------
\mainmatter
%%%----------------------------------------------------------

\chapter{Einführung}

\section{Begriffsherkunft}
Der Begriff DevOps ist eine Verbindung der Begriffe Development (Dev) und Operations (Ops). Seinen Ursprung fand er in einer Entwicklerkonferenz in Belgien, als Patrick Debois einen Namen für besagte Konferenz suchte.\\
DevOps ist also ein Zusammenschluss zwischen Softwareentwicklern und denjenigen, die die erstellte Software benutzen sollen.\cite{Peschlow.2012}

\section{Motivation}
In der traditionellen Softwareentwicklung erhält ein Softwarehaus einen Auftrag zur Erstellung einer Software. Die fertiggestellte Software wird dann dem Kunden übergeben und evtl. mit Patches versorgt. Die Arbeit der Entwickler wird dadurch nur dann honoriert, wenn diese neue Deployments leisten.\\
Auf der anderen Seite stehen diejenigen, die die Software verwenden. Diese sind an einem stabilen System interessiert, da Ausfälle in der Software zu starken Verzögerungen bei der Erledigung der Arbeit führen können. Da die Erfahrung zeigt, dass durch häufige Patches oft ein instabiles System erreicht wird, verzichten die Anwender auf häufige Updates.\\
Beide Gruppen haben also konträre Ziele, was zu Spannungen, Frust und Konflikten führen kann.\\
Aus diesem Grund kam es zur DevOps-Bewegung, die sich aus Zusammenarbeit, Automation und Prozesse zusammensetzt.\cite[S. 2 ff]{Peschlow.2012}\\

%Herkunft Begriff
%
%wo kommt DevOps her

\chapter{DevOps - Allgemein}

\section{Prinzip}


\section{Gegenüberstellung zu herkömmlicher Softwareentwicklung}


\section{Vorteile}
2016 hat die Puppet Inc eine Studie zum aktuellen Stand von DevOps veröffentlicht. Dabei nahmen 4.600 \glqq IT-Professionals\grqq an einer Online-Umfrage teil.

\section{Herausforderungen}


\section{Eignung für Projekte}

%erklären + gegenüberstellung mit herkömlicher softwareentwicklung
%
%vorteile, herausforderungen
%
%für welche projekte eignet es sich

\chapter{Einführungsbeispiel bei Projekten}

%positiv und negativ beispiele

\chapter{Fazit}

\appendix


%%%----------------------------------------------------------
\MakeBibliography
%%%----------------------------------------------------------

\listoffigures
\listoftables

\end{document}
